\documentclass[a4paper]{jsarticle}
\setlength{\topmargin}{-20.4cm}
\setlength{\oddsidemargin}{-10.4mm}
\setlength{\evensidemargin}{-10.4mm}
\setlength{\textwidth}{18cm}
\setlength{\textheight}{26cm}

\usepackage[top=15truemm,bottom=25truemm,left=20truemm,right=20truemm]{geometry}
\usepackage[latin1]{inputenc}
\usepackage{amsmath}
\usepackage{amsfonts}
\usepackage{amssymb}
\usepackage[dvipdfmx]{graphicx}
\usepackage[dvipdfmx]{color}
\usepackage{listings}
\usepackage{listings,jvlisting}
\usepackage{geometry}
\usepackage{framed}
\usepackage{color}
\usepackage[dvipdfmx]{hyperref}
\usepackage{ascmac}
\usepackage{enumerate}
\usepackage{tabularx}
\usepackage{cancel}
\usepackage{scalefnt}

\renewcommand{\figurename}{fig.}
\renewcommand{\tablename}{table }
\newcommand{\redunderline}[1]{\textcolor{BrickRed}{\underline{\textcolor{black}{#1}}}} 

\author{}
\title{CO2・温湿度計 企画書}
\date{}

\begin{document}
\maketitle
\section{コンセプト}
コンテストに提出する二酸化炭素・温湿度計の製作にあたり、コンセプトを\textgt{「誰でも製作・利用ができる作品」}とした。\par
現在のコロナ禍の影響により、自由に外出することが困難な状況となっている。
その影響を特に大きく受けているのは、飲食店や商業施設であり、
緊急事態宣言等の営業制限により、経営が困難となってお店を締めてしまう事例も少なくない。\par
また、繰り返される新型コロナウイルス感染者数の変動をみても、再度緊急事態宣言等が発出される可能性は大きく、
今後は、新型コロナウイルスと共生しながら生活していくことが不可欠であると考えられる。\par
日本政府からのガイドラインを見ると…\\
室内の温度・湿度とともにCO2濃度を誰もが視覚的に認識することができれば、施設を安心して営業・利用することができ、
今後の生活の中で欠かせないものの1つとなることは間違いない。\par
したがって、飲食店・商業施設で利用してもらえるような機能を持ち、誰でも安易に製作ができることを想定して、
設計・製作・性能検証を行った。
\section{機能}
ここで、作品の機能について、飲食店・商業施設での利用を想定し、
コンセプトから [1] 利用性:つかいやすさ、 [2] 製作性:つくりやすさ の2つの観点から決定した。
\subsection{利用性:つかいやすさ}
CO2濃度測定器は、Amazon等のネットショッピングサイトに多く出品されているが、商品レビューを見ると、センサの不具合に関した書き込みが散見された。
その中には気体中のホルムアルデヒド濃度といった全く異なる気体を計測し、その値を表示している測定器も存在している。
一般の人にはその違いを見分けることは難しく、購入・利用する際に懸念要素となる場合がある。\par
また、正確に測定できる商品であった場合でも、表示画面が大きくないものが多く、施設の利用者が自由に確認できるとは考えがたい。\par
したがって、想定される使用環境から、測定値の信頼性及び複数のユーザーが情報を素早く、正確に確認できる伝達性を機能として備えることを目標とした。
\subsection{製作性:つくりやすさ}
製作については、電子工作に詳しくない人でも製作・利用ができる必要があるため、
小学生高学年~中学生が夏休みの自由研究等で製作できる難易度程度になるように可能な限り工程を簡易化することを目標とした。\par
また、国内大手メーカーが販売しているセンサーは…と非常に高価である。
先述したネットショッピングサイトには5000円前後で出品されている商品が多いが、性能は保証されていない。\par
したがって、大手企業の製品及びネットショッピングサイトの相場価格より、目標価格を5000円~20000円程度と設定した。
\section{製作に使用したもの}
作品に使用した製品・材料、ソフトウェア、言語を以下に示す。
\subsection{材料・製品}
\begin{center}
    \begin{tabular}{|p{70mm}|p{30mm}|}
        \hline
        \multicolumn{1}{|c|}{\textgt{材料・製品名}} & \multicolumn{1}{|c|}{\textgt{価格(税抜) [円]}} \\ \hline
                                                    &                                                \\ \hline
                                                    &                                                \\ \hline
                                                    &                                                \\ \hline
                                                    &                                                \\ \hline
                                                    &                                                \\ \hline
                                                    &                                                \\ \hline
                                                    &                                                \\ \hline
                                                    & 合計 円                                        \\ \hline
    \end{tabular}
\end{center}
\subsection{ソフトウェア}
\begin{center}
    \begin{tabular}{|p{70mm}|p{30mm}|}
        \hline
        \multicolumn{1}{|c|}{\textgt{ソフトウェア名}} & \multicolumn{1}{|c|}{\textgt{用途}} \\ \hline
        Raspbian                                      & OS                                  \\ \hline
        Nginx                                         & Webサーバー                         \\ \hline
    \end{tabular}
\end{center}
\subsection{言語}
\begin{center}
    \begin{tabular}{|p{70mm}|p{30mm}|}
        \hline
        \multicolumn{1}{|c|}{\textgt{言語名}} & \multicolumn{1}{|c|}{\textgt{種類}} \\ \hline
        Python 3                              & プログラミング言語                  \\ \hline
        HTML                                  & マークアップ言語                    \\ \hline
        CSS                                   & マークアップ言語                    \\ \hline
    \end{tabular}
\end{center}
\section{設計}
\subsection{センサ}
\subsubsection{センサの選定}
使用したセンサは、インターネットを用いて、(1) youtubeやブログ等の情報量の多いモジュールであり、容易に使用できること、 
(2) 高価格ではないこと、(3) 信頼できる出力値であること の3つの観点を基準に選定した。\par
はじめに、温湿度センサは AM2320 を使用した。理由として、非常に安価であり、インターネット上の情報量が多いことが挙げられる。
信頼性については、他の温湿度計を用いて、センサからの出力値と比較したところ大きな差異はみられなかったため、
出力値の信頼性については問題ないとした。\par
また、CO2センサについては、mh-z19c を使用した。こちらも温湿度計と同様に、インターネット上の情報量が多いことに加え、
雑誌で特集されていたりとco2センサの中では比較的有名であったため採用した。
出力値の信頼性については、他のセンサを用いた比較・校正は非常に難しいが、センサの原理や実際に使用した結果に基づいて
問題ないと考える。
\subsubsection{配線}
ラズベリーパイと各センサについては以下のように配線されている。
\subsection{表示方法}
表示方法に含む機能として、情報を素早く・的確に伝達できることが挙げられる。
そこで、ラズパイをサーバとして使用し、インターネットを介して、
PC・タブレット・スマートフォンから確認できるようにした。
これにより、デバイスを問わず確認することができ、
複数の接続・表示にも対応しており、場所を問わず確認することができるため、
目的の機能であった情報の伝達機能は達成されると考える。
\subsection{プログラム}
今回の作品に使用したプログラムについて、以下に示す。
\begin{enumerate}[(1)]
    \item \textgt{センサ値の記録 (毎分)}\\
    1分ごとに、CO2センサ及び温湿度センサからの出力をcsvファイル形式で書き出す。\\
    \item \textgt{時間平均値の計算 (毎時59分)}\\
    毎時59分に、該当する時間(0~59分)のCO2量・温度・湿度の各項目の平均値を算出しcsvファイル形式で書き出す。\\
    \item \textgt{グラフの作成 (毎時59分)}\\
    毎時59分に、時間平均を出力したcsvファイルを読み込み、グラフを作成し、画像ファイルとして書き出す。\\
    \item \textgt{HTML・CSSの作成 (毎分)}\\
    1分ごとに、csvファイルとして出力された各センサの値・時間平均値の推移を示したグラフの画像ファイルを使用してHTML・CSSを書き出す。\\
\end{enumerate}
\section{試用結果}
\section{今後の展望}
\subsection{ケースの作成}
\subsection{製作マニュアルの作成}
\subsection{製作動画の撮影・公開}
\section{参考資料}
\end{document}