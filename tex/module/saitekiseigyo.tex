\documentclass[a4paper]{jsarticle}
\setlength{\topmargin}{-20.4cm}
\setlength{\oddsidemargin}{-10.4mm}
\setlength{\evensidemargin}{-10.4mm}
\setlength{\textwidth}{18cm}
\setlength{\textheight}{26cm}

\usepackage[top=15truemm,bottom=25truemm,left=20truemm,right=20truemm]{geometry}
\usepackage[latin1]{inputenc}
\usepackage{amsmath}
\usepackage{amsfonts}
\usepackage{amssymb}
\usepackage[dvipdfmx]{graphicx}
\usepackage[dvipdfmx]{color}
\usepackage{listings}
\usepackage{listings,jvlisting}
\usepackage{geometry}
\usepackage{framed}
\usepackage{color}
\usepackage[dvipdfmx]{hyperref}
\usepackage{ascmac}
\usepackage{enumerate}
\usepackage{tabularx}
\usepackage{cancel}
\usepackage{scalefnt}

\renewcommand{\figurename}{fig.}
\renewcommand{\tablename}{table }
\newcommand{\redunderline}[1]{\textcolor{BrickRed}{\underline{\textcolor{black}{#1}}}} 

\author{}
\title{最適制御システム}
\date{}

\begin{document}
\maketitle
\section{第1講 (9/27) 最適制御理論の概要}
\subsection{現代制御のアプローチ}
\subsubsection{MIMOシステム}
MIMOシステムとは、"Multi Input Multi output system(多入力多出力システム)"の略称である。
例) 飛行機の自動操縦、ドローンの操縦
対象に、単一の入力に対して単一の出力を得るシステムを SISOシステム "Single Input Multi Output system"という。
例) 自動車のアクセル操作
\begin{itembox}[l]{Point}
    \begin{center}
        入力の数と出力の数を考える
    \end{center}
\end{itembox}
\subsubsection{対象物の状況を把握するためには…}
基本的には、\textgt{直接測ることはできない}。
\begin{itemize}
    \item 直接測定することができない
    \item コストがかかりすぎる
    \item 目的の場所に取り付けられない
\end{itemize}
\subsection{Optimal Control (最適制御)}
\begin{itembox}[l]{OpOptimal Control}
    \begin{center}
        Dynamical system\\
        ↓\\
        Observaion system\\
        ↓\\
        Obeserver (状態推定器)\\
        【条件】 可観測性:制御に必要な条件は含まれているか\\
        ↓\\
        Controller (regulator)\\
        【条件】可制御性:制御できる機能はあるか\\
        ↓\\
        (ループ)\\
        \end{center}        
\end{itembox}
\begin{itembox}[l]{Cost Functional (コスト汎関数)}
    \begin{eqnarray*}
        J\left(u\right) &=& x^T\left(T\right) F x \left(T\right) \; \left(\mathrm{terminal\,Cost}\right)+ \int^T_{t_0} [x^T\left(t\right)Mx\left(t\right) \; \left(\mathrm{Error\,for\,the\,equilibrium\,states}\right)\\
        &+& u ^T N u\left(t\right)] dt \; \left(\mathrm{Control\,energy}\right) \rightarrow_u \mathrm{min.}\\
        ※&&\left\{u\left(t\right)|t_0 \geq t \geq T\right\}\\
    \end{eqnarray*}
\end{itembox}

\subsection{Optimal Control Theory}
\begin{enumerate}[(1)]
    \item Dynamic Programing (1957)
    \item Maximum principle (1958) $\rightarrow$ 少し広いがややこしい
\end{enumerate}
解き方は全く違うがたどり着く先は同じ!

\subsection{Observability}
\end{document}