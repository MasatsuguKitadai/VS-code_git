\documentclass[a4paper]{jsarticle}
\setlength{\topmargin}{-20.4cm}
\setlength{\oddsidemargin}{-10.4mm}
\setlength{\evensidemargin}{-10.4mm}
\setlength{\textwidth}{18cm}
\setlength{\textheight}{26cm}

\usepackage[top=15truemm,bottom=25truemm,left=20truemm,right=20truemm]{geometry}
\usepackage[latin1]{inputenc}
\usepackage{amsmath}
\usepackage{amsfonts}
\usepackage{amssymb}
\usepackage[dvipdfmx]{graphicx}
\usepackage[dvipdfmx]{color}
\usepackage{listings}
\usepackage{listings,jvlisting} 
\usepackage{geometry}
\usepackage{framed}
\usepackage{color}
\usepackage[dvipdfmx]{hyperref}
\usepackage{ascmac}
\usepackage{enumerate}
\usepackage{tabularx}
\usepackage{cancel}
\usepackage{scalefnt}

\renewcommand{\figurename}{fig.}
\renewcommand{\tablename}{table }
\newcommand{\redunderline}[1]{\textcolor{BrickRed}{\underline{\textcolor{black}{#1}}}} 

\hypersetup{
	colorlinks=false, % リンクに色をつけない設定
	bookmarks=true, % 以下ブックマークに関する設定
	bookmarksnumbered=true,
	pdfborder={0 0 0},
	bookmarkstype=toc
}

\lstset{
basicstyle={\ttfamily},
identifierstyle={\small},
commentstyle={\smallitshape},
keywordstyle={\small\bfseries},
ndkeywordstyle={\small},
stringstyle={\small\ttfamily},
frame={tb},
breaklines=true,
columns=[l]{fullflexible},
xrightmargin=0zw,
xleftmargin=3zw,
numberstyle={\scriptsize},
stepnumber=1,
numbersep=1zw,
lineskip=-0.5ex
}

\setcounter{tocdepth}{3}

\author{}
\title{機械力学}
\date{}

\begin{document}

\section*{数学 [1] : 解答}

\subsection*{問(1)}

\begin{itembox}[l]{POINT}
	\begin{center}
		$|\lambda - A|$ or $|A - \lambda|$ を落ち着いて因数分解する!!
	\end{center}
\end{itembox}

\begin{eqnarray*}
	|A - \lambda|
	=
	\begin{vmatrix}
		2-\lambda & -1        & -1        \\
		-1        & 2-\lambda & -1        \\
		-1        & -1        & 2-\lambda \\
	\end{vmatrix}
	=
	-\lambda(\lambda-3)^2
\end{eqnarray*}

\noindent

特性方程式は,

\begin{eqnarray*}
	-\lambda(\lambda^2 -3) = 0
\end{eqnarray*}

\noindent
これを解くと,$\lambda = 0, 3$.\\
しがって,求める固有値は \underline{$\lambda = 0, 3$.}

\subsection*{(2)}

\subsubsection*{基底とは}

\begin{itembox}[l]{定義}
	$W$を部分空間とするとき,$W$内のベクトルの系$\left(v_1,v_2,\cdots,v_n\right)$に対して,
	以下の2つの条件が成立するとき$\left(v_1,v_2,\cdots,v_n\right)$を$W$の\textgt{基底}であるという.
	\begin{enumerate}[(1)]
		\item $W$の任意のベクトルが,$\left(v_1,v_2,\cdots,v_n\right)$の一次結合で表される
		\item $\left(v_1,v_2,\cdots,v_n\right)$は一次独立である
	\end{enumerate}
\end{itembox}
条件より,$W$の任意のベクトル$\vec{v}$は
\begin{eqnarray*}
	\vec{v}=x_1v_1+x_2v_2+\cdots+x_nv_n
\end{eqnarray*}
とただ一通りに表すことができる.\\

\begin{itembox}[l]{POINT}
	\begin{enumerate}[(1)]
		\item $|\lambda - A| = 0$ にそれぞれの固有値$\lambda$を代入して整理.
		\item 任意定数を与えて,連立方程式を解く.
		\item 好きな任意定数を固定して,その他の任意定数との関係を調べる.
		\item [※] 解が定まらないときは,更に追加で任意定数を固定する!!
		\item 出てきた値がベクトル空間の基底になる.
	\end{enumerate}
\end{itembox}

\subsubsection*{(i) $\lambda = 0$ のとき}

\begin{eqnarray*}
	\begin{bmatrix}
		2-\lambda & -1        & -1        \\
		-1        & 2-\lambda & -1        \\
		-1        & -1        & 2-\lambda \\
	\end{bmatrix}
	\rightarrow
	\begin{bmatrix}
		2-0 & -1  & -1  \\
		-1  & 2-0 & -1  \\
		-1  & -1  & 2-0 \\
	\end{bmatrix}
	\rightarrow
	\begin{bmatrix}
		2  & -1 & -1 \\
		-1 & 2  & -1 \\
		-1 & -1 & 2  \\
	\end{bmatrix}
	\rightarrow
	\begin{bmatrix}
		0  & -3 & 3  \\
		0  & 3  & -3 \\
		-1 & -1 & 2  \\
	\end{bmatrix}
	\rightarrow
	\begin{bmatrix}
		-1 & -1 & 2  \\
		0  & 1  & -1 \\
		0  & 0  & 0  \\
	\end{bmatrix}\\
\end{eqnarray*}

任意定数を$C_1$,$C_2$,$C_3$とすると

\begin{eqnarray*}
	-C_1 -C_2 + 2C_3 &=& 0\\
	C_2 -C_3 &=& 0
\end{eqnarray*}

\begin{eqnarray*}
	C_1 = C_2 = C_3\\
\end{eqnarray*}

\noindent
ここで,仮に$C_1 = 1$とすると,$C_2 = C_3 = 1$ となる.\\

\noindent
したがって,$\lambda = 0$ における基底は,
\underline{$
		\begin{bmatrix}
			1 \\
			1 \\
			1 \\
		\end{bmatrix}
	$}

\subsubsection*{(ii) $\lambda = 3$ のとき}

\begin{eqnarray*}
	\begin{bmatrix}
		2-\lambda & -1        & -1        \\
		-1        & 2-\lambda & -1        \\
		-1        & -1        & 2-\lambda \\
	\end{bmatrix}
	\rightarrow
	\begin{bmatrix}
		2-3 & -1  & -1  \\
		-1  & 2-3 & -1  \\
		-1  & -1  & 2-3 \\
	\end{bmatrix}
	\rightarrow
	\begin{bmatrix}
		-1 & -1 & -1 \\
		-1 & -1 & -1 \\
		-1 & -1 & -1 \\
	\end{bmatrix}
	\rightarrow
	\begin{bmatrix}
		1 & 1 & 1 \\
		0 & 0 & 0 \\
		0 & 0 & 0 \\
	\end{bmatrix} \\
\end{eqnarray*}

(1) と同様に任意定数 $C_1$,$C_2$,$C_3$ とすると,

\begin{eqnarray*}
	C_1 + C_2 + C_3 = 0
\end{eqnarray*}

$C_1 = 1$ としたとき,

\begin{eqnarray*}
	C_2 + C_3 = -1
\end{eqnarray*}

$C_2 = -1$ とすると,

\begin{eqnarray*}
	C_3 = 0
\end{eqnarray*}\\

このとき,$\lambda = 3$ における基底は
\underline{$
		\begin{bmatrix}
			1  \\
			-1 \\
			0  \\
		\end{bmatrix}
	$}

また,$C_3 = -1$ とすると,

\begin{eqnarray*}
	C_2 = 0
\end{eqnarray*}

このとき,$\lambda = 3$ における基底は
\underline{$
		\begin{bmatrix}
			1  \\
			0  \\
			-1 \\
		\end{bmatrix}
	$}

\section*{問(3)}

\subsubsection*{行列の対角化}

問(2)の基底のベクトルを正規直行化すると,$	\begin{bmatrix}
		1/\sqrt{3} \\
		1/\sqrt{3} \\
		1/\sqrt{3} \\
	\end{bmatrix} $
,$
	\begin{bmatrix}
		1/\sqrt{2}  \\
		-1/\sqrt{2} \\
		0           \\
	\end{bmatrix}
$
,$
	\begin{bmatrix}
		1/\sqrt{6}  \\
		1/\sqrt{6}  \\
		-2/\sqrt{6} \\
	\end{bmatrix}\\
$\\

以上の正規直交基底より,
$
	P =
	\begin{bmatrix}
		1/\sqrt{3} & 1/\sqrt{2}  & 1/\sqrt{6}  \\
		1/\sqrt{3} & -1/\sqrt{2} & 1/\sqrt{6}  \\
		1/\sqrt{3} & 0           & -2/\sqrt{6} \\
	\end{bmatrix}\\
$\\

これを用いて対角化すると,
$P^tAP =
	\begin{bmatrix}
		0 & 0 & 0 \\
		0 & 3 & 0 \\
		0 & 0 & 3 \\
	\end{bmatrix}
$

\end{document}