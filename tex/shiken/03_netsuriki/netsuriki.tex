\documentclass[a4paper]{jsarticle}
\setlength{\topmargin}{-20.4cm}
\setlength{\oddsidemargin}{-10.4mm}
\setlength{\evensidemargin}{-10.4mm}
\setlength{\textwidth}{18cm}
\setlength{\textheight}{26cm}

\usepackage[top=15truemm,bottom=25truemm,left=20truemm,right=20truemm]{geometry}
\usepackage[latin1]{inputenc}
\usepackage{amsmath}
\usepackage{amsfonts}
\usepackage{amssymb}
\usepackage[dvipdfmx]{graphicx}
\usepackage[dvipdfmx]{color}
\usepackage{listings}
\usepackage{listings,jvlisting}
\usepackage{geometry}
\usepackage{framed}
\usepackage{color}
\usepackage[dvipdfmx]{hyperref}
\usepackage{ascmac}
\usepackage{enumerate}
\usepackage{tabularx}
\usepackage{cancel}
\usepackage{scalefnt}

\renewcommand{\figurename}{fig.}
\renewcommand{\tablename}{table }
\newcommand{\redunderline}[1]{\textcolor{BrickRed}{\underline{\textcolor{black}{#1}}}} 

\hypersetup{
	colorlinks=false, % リンクに色をつけない設定
	bookmarks=true, % 以下ブックマークに関する設定
	bookmarksnumbered=true,
	pdfborder={0 0 0},
	bookmarkstype=toc
}

\lstset{
basicstyle={\ttfamily},
identifierstyle={\small},
commentstyle={\smallitshape},
keywordstyle={\small\bfseries},
ndkeywordstyle={\small},
stringstyle={\small\ttfamily},
frame={tb},
breaklines=true,
columns=[l]{fullflexible},
xrightmargin=0zw,
xleftmargin=3zw,
numberstyle={\scriptsize},
stepnumber=1,
numbersep=1zw,
lineskip=-0.5ex
}

\setcounter{tocdepth}{3}

\author{}
\title{熱力学}
\date{}

\begin{document}
\maketitle
\tableofcontents
\newpage

\section{熱力学第一法則}
\begin{screen}
    \begin{center}
        \textgt{熱}と\textgt{仕事}は本質的に同種のエネルギーであり,それらは互いに\textgt{変換可能}である
    \end{center}
\end{screen}
\subsection{閉じた系}
物質の出入りのない系のことを\textgt{閉じた系}という.\quad 例) ピストン-シリンダ系
\begin{itembox}[l]{閉じた系の熱力学第一法則}
    \begin{eqnarray*}
        Q&=&\Delta U+W\\
        Q&:&受熱量\\
        \Delta U&:&内部エネルギの変化量\\
        W&:&絶対仕事 \; (気体のした仕事)\\
    \end{eqnarray*}
\end{itembox}
\subsection{内部エネルギ}
個々の分子は力学的エネルギ(位置エネルギ+運動エネルギ)を持っている.\\
それを巨視的に見たときに分子の持つ\textgt{内部エネルギ}と呼ぶ.
\begin{itembox}[l]{内部エネルギの変化量}
    \begin{eqnarray*}
        \Delta U&=&mC_v\Delta T\\
    \end{eqnarray*}
\end{itembox}
初期状態を$\left(P,v,T\right)$とする.\\
※ $v$は比容積
\subsection{開いた系}
物質の出入りがある系のことを\textgt{開いた系}という.\quad 例) タービン\\
また,このとき取り出される仕事を\textgt{工業仕事}という.
\begin{itembox}[l]{開いた系の熱力学第一法則}
    \begin{eqnarray*}
        Q&=&\Delta H+W_t\\
        Q&:&受熱量\\
        \Delta H&:&エンタルピの変化量\\
        W_t&:&工業仕事 \; (気体のした仕事)\\
    \end{eqnarray*}
\end{itembox}
\subsection{エンタルピ}
内部エネルギーと流動エネルギを合わせたものをエンタルピと定義される.
\begin{itembox}[l]{Point}
    \begin{center}
        閉じた系における\textgt{等圧過程での受熱量(放熱量)}のこと!!
    \end{center}
\end{itembox}
\begin{itembox}[l]{エンタルピ}
    \begin{eqnarray*}
        H=U+W\\
    \end{eqnarray*}
\end{itembox}
\begin{itembox}[l]{エンタルピの変化量}
    \begin{eqnarray*}
        \Delta H&=&mC_p\Delta T\\
    \end{eqnarray*}
\end{itembox}
\section{理想気体}
\begin{itembox}[l]{状態方程式}
    \begin{eqnarray*}
        PV&=&mRT\\
        Pv&=&RT\quad (単位質量基準の場合)\\
        ※v&:&比容積\\
    \end{eqnarray*}
\end{itembox}
\begin{itembox}[l]{マイヤーの法則}
    \begin{eqnarray*}
        C_p&=&C_v+R\\
        C_p&:&定圧比熱\\
        C_v&:&定容比熱\\
    \end{eqnarray*}
\end{itembox}
\section{熱力学第二法則}
\begin{screen}
    \begin{center}
        熱は高いほうから低いほうへと移動するという\textgt{不可逆現象}のことを指す.
    \end{center}
\end{screen}
\subsection{仕事}
\begin{itembox}[l]{仕事}
    \begin{eqnarray*}
        W&=&Q_H-Q_L\\
        W&=&\displaystyle \int PdV\quad(閉じた系の絶対仕事)\\
        W_t&=&\displaystyle \int VdP\quad(開いた系の工業仕事) \\
    \end{eqnarray*}
\end{itembox}
\begin{itembox}[l]{正味の仕事}
    \begin{eqnarray*}
        W_{net} &=& (膨張過程で\;\textgt{した仕事}) - (圧縮過程で\;\textgt{された仕事})\\
        &=& (膨張過程で\;\textgt{した仕事}) + (圧縮過程で\;\textgt{した仕事})\\
        &=& (膨張過程で\;\textgt{された仕事}) + (圧縮過程で\;\textgt{された仕事})\\
    \end{eqnarray*}
\end{itembox}
\subsection{熱効率}
\begin{itembox}[l]{Point}
    \begin{center}
        基本は,\textgt{受熱量$Q_{in}$}と\textgt{放熱量$O_{out}$}で考える\\
        ※ 等圧変化を含むサイクルの場合、正味の仕事$W_{net}$は加熱過程にも含まれるため、計算が複雑になる
    \end{center}
\end{itembox}
\begin{itembox}[l]{熱効率}
    \begin{eqnarray*}
        \eta_{th}=\dfrac{W_{net}}{Q_H}=1-\dfrac{Q_L}{Q_H}\\
    \end{eqnarray*}
\end{itembox}
\section{エントロピ}
2つの状態間を\textgt{不可逆的に変化}させたとき,
その変化の経路に無関係に一定となる値(\textgt{状態量})のこと.\\
また,その変化の不可逆性の大きさを示す.可逆断熱過程では,エントロピーの変化は\textgt{ゼロ}になる.
\begin{itembox}[l]{エントロピ}
    \begin{eqnarray*}
        \displaystyle\int \frac{dQ}{T}=const\\
    \end{eqnarray*}
\end{itembox}
\section{加熱・放熱}
\begin{itembox}[l]{Point}
    \begin{center}
        T-S線図をすぐにイメージできるかがカギ\\
        \textgt{右肩上がりの線}(比例)が描ける\\
        $C_p > C_v$ より,傾きは \textgt{定積変化 $>$ 定容変化}\\
    \end{center}
\end{itembox}
\begin{itembox}[l]{加熱・放熱}
    \begin{enumerate}[(1)]
        \item 等容加熱
              \begin{center}
                  体積 → \textgt{一定} , 圧力 → \textgt{増加} , 温度 → \textgt{増加} , エントロピ → \textgt{増加}
              \end{center}
        \item 等容放熱
              \begin{center}
                  体積 → \textgt{一定} , 圧力 → \textgt{減少} , 温度 → \textgt{減少} , エントロピ → \textgt{減少}
              \end{center}
        \item 等圧加熱
              \begin{center}
                  体積 → \textgt{増加} , 圧力 → \textgt{一定} , 温度 → \textgt{増加} , エントロピ → \textgt{増加}
              \end{center}
        \item 等圧放熱
              \begin{center}
                  体積 → \textgt{減少} , 圧力 → \textgt{一定} , 温度 → \textgt{減少} , エントロピ → \textgt{減少}\\
              \end{center}
    \end{enumerate}
\end{itembox}
\subsection{等容過程}
\begin{itembox}[l]{Point}
    \begin{eqnarray*}
        \frac{P}{T}&=&const\\
        Q&=&\Delta U = mC_v\Delta T\\
        \\
        \Delta s&=&\int \frac{dQ}{T} = \int \frac{dU}{T} = C_v\int\frac{1}{T}dT\\
    \end{eqnarray*}
\end{itembox}
\subsection{等圧過程}
\begin{itembox}[l]{Point}
    \begin{eqnarray*}
        \frac{V}{T}&=&const\\
        Q&=&\Delta U + W = mC_p\Delta T \; \left(= \Delta H\right)\\
        \\
        \Delta s&=&\int \frac{dQ}{T} = \int \frac{dH}{T} = C_p\int\frac{1}{T}dT\\
    \end{eqnarray*}
\end{itembox}
\section{膨張・圧縮}
\begin{itembox}[l]{Point}
    \begin{center}
        P-V線図をすぐにイメージできるかがカギ\\
        \textgt{右肩下がりの線}(反比例)が描ける\\
        体積の変化量が等しいとき,$C_p > C_v$ より,傾きは \textgt{断熱変化 $>$ 等温変化}\\
    \end{center}
\end{itembox}
\begin{itembox}[l]{膨張・圧縮}
    \begin{enumerate}[(1)]
        \item 等温膨張
              \begin{center}
                  体積 → \textgt{増加} , 圧力 → \textgt{減少} , 温度 → \textgt{一定} , エントロピ → \textgt{増加}
              \end{center}
        \item 等温圧縮
              \begin{center}
                  体積 → \textgt{減少} , 圧力 → \textgt{増加} , 温度 → \textgt{一定} , エントロピ → \textgt{減少}
              \end{center}
        \item 断熱膨張
              \begin{center}
                  体積 → \textgt{増加} , 圧力 → \textgt{減少} , 温度 → \textgt{減少}  , エントロピ → \textgt{一定}
              \end{center}
        \item 断熱圧縮
              \begin{center}
                  体積 → \textgt{減少} , 圧力 → \textgt{増加} , 温度 → \textgt{増加}  , エントロピ → \textgt{一定}
              \end{center}
    \end{enumerate}
\end{itembox}
\subsection{等温過程}
\begin{itembox}[l]{Point}
    \begin{eqnarray*}
        PV&=&const\\
        Q&=&W\\
        \Delta S
        &=& \int \frac{dQ}{T}\\
        &=& \frac{1}{T}\int Pdv\;(閉じた系)\\
        &=& \frac{1}{T}\int Vdp\;(開いた系)\\
    \end{eqnarray*}
\end{itembox}
\subsection{断熱(等エントロピー)過程}
\begin{itembox}[l]{Point}
    \begin{eqnarray*}
        PV^\kappa&=&const\\
        \Delta U&=&- W\\
        \Delta s&=&0\\
    \end{eqnarray*}
\end{itembox}
\subsection{ポアソンの式とその導出}
\begin{itembox}[l]{ポアソンの式}
    \begin{eqnarray*}
        PV^\kappa=const\\
    \end{eqnarray*}
\end{itembox}
熱力学第一法則より,断熱過程では
\begin{eqnarray*}
    0
    &=&du+Pdv\\
    &=&C_vdT+Pdv\\
\end{eqnarray*}
両辺に気体定数$R$をかけて,
\begin{eqnarray*}
    0=C_vRdT+RPdv
\end{eqnarray*}
理想気体の式$Pv=RT$と比熱の関係式,マイヤーの式$C_v+R=C_p$より
\begin{eqnarray*}
    0&=&C_vd\left(Pv\right)+\left(C_p-C_v\right)Pdv\\
    &=&C_v\left(Pdv+vdP\right)+\left(C_p-C_v\right)Pdv\\
    &=&C_vvdP+C_pPdv\\
\end{eqnarray*}
両辺を$C_vPv$で割って
\begin{eqnarray*}
    0=\frac{dP}{P}+\kappa\frac{du}{v}\\
\end{eqnarray*}
両辺を積分すると,
\begin{eqnarray*}
    const&=&\ln P+\kappa\ln v\\
    &=&Pv^\kappa\\
\end{eqnarray*}
\section{サイクル}
あくまで,理論上のサイクルでありすべて\textgt{閉じた系}と仮定している.
\subsection{カルノーサイクル}
\begin{itembox}[l]{過程}
    \begin{center}
        \textgt{(1) 断熱圧縮}\quad → \quad \textgt{(2) 等温膨張} \quad → \quad \textgt{(3) 断熱膨張} \quad → \quad \textgt{(4) 等温圧縮}
    \end{center}
\end{itembox}
受熱量をすべて仕事に変換できる等温変化を用いるため,サイクルの中で最も高い熱効率となる.
\begin{itembox}[l]{熱効率}
    \begin{eqnarray*}
        \eta_{thc}=1-\frac{T_2}{T_1}\\
    \end{eqnarray*}
\end{itembox}
\subsection{オットーサイクル}
\begin{itembox}[l]{過程}
    \begin{center}
        \textgt{(1) 断熱圧縮}\quad → \quad \textgt{(2) 等容加熱} \quad → \quad \textgt{(3) 断熱膨張} \quad → \quad \textgt{(4) 等容放熱}
    \end{center}
\end{itembox}
\begin{itembox}[l]{圧縮率}
    \begin{eqnarray*}
        \varepsilon=\frac{V_1}{V_2}=\frac{V_3}{V_4}> 1\\
    \end{eqnarray*}
\end{itembox}
\begin{itembox}[l]{熱効率}
    \begin{eqnarray*}
        \eta_{tho}=1-\frac{1}{\varepsilon^{\kappa-1}}\\
    \end{eqnarray*}
\end{itembox}
ガソリンエンジンの基本サイクル.
\subsection{ディーゼルサイクル}
\begin{itembox}[l]{過程}
    \begin{center}
        \textgt{(1) 断熱圧縮}\quad → \quad \textgt{(2) 等圧加熱} \quad → \quad \textgt{(3) 断熱膨張} \quad → \quad \textgt{(4) 等容放熱}
    \end{center}
\end{itembox}
\begin{itembox}[l]{締切比 (等圧膨張比)}
    \begin{eqnarray*}
        \sigma=\frac{V_3}{V_2}>1\\
    \end{eqnarray*}
\end{itembox}
ディーゼルエンジンのサイクル.
\subsection{サバテサイクル}
\begin{itembox}[l]{過程}
    \begin{center}
        \textgt{(1) 断熱圧縮}\quad → \quad \textgt{(2) 等容加熱}\quad → \quad \textgt{(3) 等圧加熱} \quad → \quad \textgt{($3'$) 断熱膨張} \quad → \quad \textgt{(4) 等容放熱}
    \end{center}
\end{itembox}
\begin{itembox}[l]{圧力比}
    \begin{eqnarray*}
        \alpha=\frac{P_3}{P_2}>1\\
    \end{eqnarray*}
\end{itembox}
高速ディーゼルサイクル.受熱過程が等容変化と等圧変化の組み合わせになっている.
\subsection{ブレイトンサイクル}
\begin{itembox}[l]{過程}
    \begin{center}
        \textgt{(1) 断熱圧縮}\quad → \quad \textgt{(2) 等圧加熱} \quad → \quad \textgt{(3) 断熱膨張} \quad → \quad \textgt{(4) 等圧放熱}
    \end{center}
\end{itembox}
ガスタービンやジェットエンジンの基本サイクル.\\
※ 閉じた系と仮定している
\begin{itembox}[l]{断熱効率}
    \begin{eqnarray*}
        \eta &=& \dfrac{h_{2s}-h_1}{h_{2a}-h_1} <1 \quad (圧縮機)\\
        \eta &=& \dfrac{h_3-h_{4a}}{h_3-h_{4s}} <1 \quad (タービン)\\
    \end{eqnarray*}
\end{itembox}
\begin{itembox}[l]{逆仕事比}
    \begin{eqnarray*}
        r_bw &=& \frac{w_c}{w_t}\\
        w_c&:&圧縮機からされた仕事\\
        w_t&:&タービンに下仕事\\
    \end{eqnarray*}
\end{itembox}
\section{再生ブレイトンサイクル}
\section{乾き度}
\begin{itembox}[l]{Point}
    \begin{center}
        単位質量あたりに含まれている蒸気の質量 → \textgt{乾き度}
    \end{center}
\end{itembox}
\\
蒸気中の気相部分と液相部分の重量割合のことを\textgt{乾き度}という.\\
$1\mathrm{kg}$の湿り蒸気が$x\mathrm{kg}$の蒸気と$\left(1-x\right)\mathrm{kg}$の液体から成るとき,
$x$を\textgt{乾き度},$\left(1-x\right)$を\textgt{湿り度}という.

\subsection{湿り蒸気}
飽和蒸気と飽和水が共存している状態のこと.

\end{document}