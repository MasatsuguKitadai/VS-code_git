\documentclass[a4paper]{jsarticle}
\setlength{\topmargin}{-20.4cm}
\setlength{\oddsidemargin}{-10.4mm}
\setlength{\evensidemargin}{-10.4mm}
\setlength{\textwidth}{18cm}
\setlength{\textheight}{26cm}

\usepackage[top=15truemm,bottom=25truemm,left=20truemm,right=20truemm]{geometry}
\usepackage[latin1]{inputenc}
\usepackage{amsmath}
\usepackage{amsfonts}
\usepackage{amssymb}
\usepackage[dvipdfmx]{graphicx}
\usepackage{listings}
\usepackage{listings,jvlisting}
\usepackage{geometry}
\usepackage{framed}
\usepackage{color}
\usepackage[dvipdfmx]{hyperref}
\usepackage{ascmac}
\usepackage{enumerate}
\usepackage{tabularx}

\hypersetup{
	colorlinks=false, % リンクに色をつけない設定
	bookmarks=true, % 以下ブックマークに関する設定
	bookmarksnumbered=true,
	pdfborder={0 0 0},
	bookmarkstype=toc
}

\lstset{
basicstyle={\ttfamily},
identifierstyle={\small},
commentstyle={\smallitshape},
keywordstyle={\small\bfseries},
ndkeywordstyle={\small},
stringstyle={\small\ttfamily},
frame={tb},
breaklines=true,
columns=[l]{fullflexible},
xrightmargin=0zw,
xleftmargin=3zw,
numberstyle={\scriptsize},
stepnumber=1,
numbersep=1zw,
lineskip=-0.5ex
}



\author{}
\title{数学}
\date{}

\begin{document}
\maketitle

\section{線形代数}
\subsection{行列式}
\subsubsection{余因子行列とクラメールの公式}
$n$次正方行列$A=\left[a_{ij}\right]$の第$i$行と第$j$列を取り除いて得られる$n-1$次正方行列を$A_{ij}$と書く.
\begin{itembox}[l]{余因子展開}
    \begin{enumerate}[(1)]
        \item 第$j$列に関する余因子展開\\
              $\left|A\right|=(-1)^{1+j}a_{1j}\left|A_{1j}\right|+(-1)^{2+j}a_{2j}\left|A_{2j}\right|+ \dots +(-1)^{n+j}a_{nj}\left|A_{nj}\right|$
        \item 第$i$行に関する余因子展開\\
              $\left|A\right|=(-1)^{i+1}a_{1j}\left|A_{i1}\right|+(-1)^{i+2}a_{i2}\left|A_{i2}\right|+ \dots +(-1)^{i+n}a_{in}\left|A_{in}\right|$
    \end{enumerate}
\end{itembox}
\begin{itembox}[l]{余因子行列}
    $n$次正方行列$\left|a_{ij}\right|$に対し,
    \begin{center}
        $\check{a}_{ij}=\left(-1\right)^{i+j}\left|A_{ji}\right|$
    \end{center}
    さらに,以下のようにおき,$A$の余因子行列という.
    \begin{center}
        $\tilde{A}=\left[\check{a}_{ij}\right]$
    \end{center}
\end{itembox}
\begin{itembox}[l]{定理3.4.1}
    正方行列$A$の余因子行列を$\tilde{A}$とすると,以下の関係が成立する.
    \begin{center}
        $A\tilde{A}=\tilde{A}A=dE\left (d=\rm{det}\left(A\right)\right)$
    \end{center}
\end{itembox}
\begin{itembox}[l]{クラメールの公式}
\end{itembox}
\subsection{連立一次方程式,基本変形}
\subsubsection{連立一次方程式の消去法による解法}
\begin{enumerate}[(i)]
    \item 連立一次方程式より拡大係数行列を抽出(行列のデータ化)
    \item 抽出した拡大係数行列を簡約化する(未知数の整理)
    \item 簡約化の結果を連立一次方程式に還元し解を作成する
\end{enumerate}
\subsubsection{解が不定の場合}
「1式に1未知数」という形は,一般には成り立たない.そこで,「1式に1未知数」に近い形に整理したものが\textgt{階段行列(=簡約行列)}である.
一般の連立1次方程式の場合,未知数,方程式の本数,任意定数の間には以下の関係が成り立つ.
\begin{center}
    (未知数の個数) = (有効な方程式の数) + (任意定数の数)
\end{center}
\subsection{固有値と固有ベクトルの計算}
\subsubsection{固有値と固有ベクトル}
\begin{itembox}[l]{定義}
    $n$次正方行列$A$とスカラー$\lambda$に対し,
    \begin{center}
        $Ax=\lambda x$
    \end{center}
    となる\textgt{零ベクトル$o$ではないベクトル$x$}が存在するとき,$\lambda$を$A$の固有値といい,\\
    $\lambda$に対し上の条件を満たす$o$ではないベクトル$x$を$A$の固有ベクトルという.
    また,固有値を求める際には,$|\lambda E-A|=0$を解けば良い.
\end{itembox}
\subsubsection{行列の対角化}
正方行列$A$が与えられたとき,$B=P^{-1}AP$が対角行列になるような生息行列$P$と対角行列$B$を
求めることを行列$A$の対角化という.
\begin{itembox}[l]{固有多項式}
    正方行列$A$に対し,以下の多項式$g_A\left(t\right)$を$A$の固有多項式という.
    \begin{center}
        $g_A\left(t\right) = \left|tE-A\right| $
    \end{center}
\end{itembox}
\subsubsection{対角化可能性}
正方行列$A$は常に対角化されるとは限らない.
\begin{itembox}[l]{対角化の条件}
    $A$が$N$次の正方行列のとき,「$A$が対角化可能」とは
    $A$が$N$個の独立な固有ベクトルを持つことと同値である.
    (固有値の重複度と固有空間の次元が一致している)
\end{itembox}
\subsubsection{対角化の手順}
対角化の作業は以下の手順で行われる.
\begin{enumerate}[(i)]
    \item 固有値の計算
    \item 各固有値に対する固有ベクトルの計算
    \item 上の手順で得られた(一次独立な)固有ベクトルの組を並べてできた行列を$P$としたとき,\\
          この$P$は正則行列であり,$P^{-1}AP$は対角行列となる.
\end{enumerate}
\newpage
\section{微積分}
\subsection{関数のべき級数展開}
関数のべき級数展開とは,関数を多項式で近似していくことを指す.
工学分野では,必要不可欠なツールであり,出題内容として(1) 関数を展開す問題
(2) 近似・誤差の評価という2種類が主なものである.
\subsubsection{Taylor展開}
\begin{itembox}[l]{Taylor展開}
    \begin{center}
        $f\left(x\right)=f\left(0\right)+f^{\left(1\right)}\left(0\right)x
            +\dfrac{f^{\left(2\right)}\left(0\right)}{2}x^2
            +\dfrac{f^{\left(3\right)}\left(0\right)}{3}x^3
            +\dots
            +\dfrac{f^{\left(n\right)}\left(0\right)}{n}x^n$
    \end{center}
\end{itembox}
\subsection{1変数の積分}
\subsubsection{偶関数と奇関数}

\subsection{多変数関数}
\subsubsection{偏微分}
多変数関数の微分 → 偏微分
\subsubsection{多変数関数の極値}
\begin{itembox}[l]{定理 4.3.3 極値を持つ必要条件}
    $f\left(x,y\right)$ が $\left(a,b\right)$で極値をとるならば,
    $f_x\left(a,b\right)=f_y\left(a,b\right)=0$ である.
\end{itembox}
\begin{itembox}[l]{定理 極値の判定}
    $f\left(x,y\right)$は$C^2$の関数であり,点$\left(a,b\right)$において
    $f_x\left(a,b\right)=f_y\left(a, b\right)=0$であるとする.\\
    判別式を,$D=f_{xx}\left(a,b\right)f_{yy}\left(a,b\right)-{f_{xy}(a,b)}^2$と定義する.
    \begin{enumerate}[(1)]
        \item $D>0$とする.\\
              $f_{xx}\left(a,b\right)>0$ならば,$f$は点$\left(a,b\right)$で極小値をとる.\\
              $f_{xx}\left(a,b\right)<0$ならば,$f$は点$\left(a,b\right)$で極大値をとる.
        \item $D<0$ならば,$f$は点$\left(a,b\right)$で極値をとらない.
    \end{enumerate}
\end{itembox}
\subsubsection{接平面の方程式}
関数のグラフ$z=f\left(x,y\right)$は空間内の局面を定める。この曲面上の$p(x_0,y_0,z_0)$における接平面の方程式は、
\begin{itembox}[l]{接平面の方程式}
    \begin{center}
        $z-z_0=f_x\left(x_0,y_0\right)\left(x-x_0\right)+f_y\left(x_0,y_0\right)\left(y-y_0\right)$
    \end{center}
\end{itembox}
により与えられる。
\subsubsection{接線の方程式と陰関数の定理}
\begin{itembox}[l]{定理 4.4.1 陰関数の定理}
    $f\left(x,y\right)$が$C^1$級の関数で,$f_x\left(a,b\right)=0$, $f_y\left(a,b\right)\neq0$ならば,\\
    $a$を含む開区間で定義された$f(x,y)=0$の陰関数$y=\phi\left(x\right)$で,$\phi\left(a\right)=b$となるものが存在する.\\
    このとき,$y=\phi\left(x\right)$は微分可能で,次の式が成り立つ.
    \begin{center}
        $\phi'\left(x\right)=\dfrac{f_x\left(x,\phi\left(x\right)\right)}{f_y\left(x,\phi\left(x\right)\right)}$,
        すなわち,$\dfrac{dy}{dx}=-\dfrac{f_x\left(x,y\right)}{f_y\left(x,y\right)}$
    \end{center}
\end{itembox}
\newpage
\section{材料力学}
\subsection{重要公式}
\begin{itembox}[l]{引張・圧縮}
    \begin{center}
        \begin{eqnarray*}
            \sigma&=&\dfrac{P}{A}\\
            \varepsilon&=&\dfrac{\lambda}{L}\\
            \sigma&=&E\varepsilon\\
            P&=&\dfrac{AE}{l}\lambda=AE\varepsilon
        \end{eqnarray*}
        $AE$  :引張剛性
    \end{center}
\end{itembox}
\begin{itembox}[l]{ねじり}
    \begin{center}
        \begin{eqnarray*}
            \tau\left(r\right)&=&\dfrac{T}{I_p} r\\
            \gamma\left(r\right)&=&\dfrac{\lambda\left(r\right)}{L}=\dfrac{\varphi}{L}r=\theta r\\
            \tau\left(r\right)&=&G\gamma\left(r\right)=G\theta r\\
            T&=&\dfrac{I_pG}{L}\varphi
        \end{eqnarray*}
        $\theta=\dfrac{\varphi}{L}$ :比ねじれ角\\
        $I_pG$  :ねじれ剛性
    \end{center}
\end{itembox}

\end{document}