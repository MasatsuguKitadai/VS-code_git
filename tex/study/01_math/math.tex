\documentclass[a4paper]{jsarticle}
\setlength{\topmargin}{-20.4cm}
\setlength{\oddsidemargin}{-10.4mm}
\setlength{\evensidemargin}{-10.4mm}
\setlength{\textwidth}{18cm}
\setlength{\textheight}{26cm}

\usepackage[top=15truemm,bottom=25truemm,left=20truemm,right=20truemm]{geometry}
\usepackage[latin1]{inputenc}
\usepackage{amsmath}
\usepackage{amsfonts}
\usepackage{amssymb}
\usepackage[dvipdfmx]{graphicx}
\usepackage{listings}
\usepackage{listings,jvlisting}
\usepackage{geometry}
\usepackage{framed}
\usepackage{color}
\usepackage[dvipdfmx]{hyperref}
\usepackage{ascmac}

\hypersetup{
	colorlinks=false, % リンクに色をつけない設定
	bookmarks=true, % 以下ブックマークに関する設定
	bookmarksnumbered=true,
	pdfborder={0 0 0},
	bookmarkstype=toc
}

\lstset{
basicstyle={\ttfamily},
identifierstyle={\small},
commentstyle={\smallitshape},
keywordstyle={\small\bfseries},
ndkeywordstyle={\small},
stringstyle={\small\ttfamily},
frame={tb},
breaklines=true,
columns=[l]{fullflexible},
xrightmargin=0zw,
xleftmargin=3zw,
numberstyle={\scriptsize},
stepnumber=1,
numbersep=1zw,
lineskip=-0.5ex
}



\author{}
\title{数学}
\date{}

\begin{document}
\maketitle

\section{線形代数}
\subsection{行列式}
\subsubsection{余因子行列とクラメールの公式 P.54}
$n$次正方行列$A=\left[a_{ij}\right]$の第$i$行と第$j$列を取り除いて得られる$n-1$次正方行列を$A_{ij}$と書く。
\begin{itembox}[l]{余因子展開}
    \begin{enumerate}
        \item 第$j$列に関する余因子展開\\
              $\left|A\right|=(-1)^{1+j}a_{1j}\left|A_{1j}\right|+(-1)^{2+j}a_{2j}\left|A_{2j}\right|+ \dots +(-1)^{n+j}a_{nj}\left|A_{nj}\right|$
        \item 第$i$行に関する余因子展開\\
              $\left|A\right|=(-1)^{i+1}a_{1j}\left|A_{i1}\right|+(-1)^{i+2}a_{i2}\left|A_{i2}\right|+ \dots +(-1)^{i+n}a_{in}\left|A_{in}\right|$
    \end{enumerate}
\end{itembox}
\begin{itembox}[l]{余因子行列}
    $n$次正方行列$\left|a_{ij}\right|$に対し、
    \begin{center}
        $\check{a}_{ij}=\left(-1\right)^{i+j}\left|A_{ji}\right|$
    \end{center}
    さらに、以下のようにおき、$A$の余因子行列という。
    \begin{center}
        $\tilde{A}=\left[\check{a}_{ij}\right]$
    \end{center}
\end{itembox}
\begin{itembox}[l]{定理3.4.1}
    正方行列$A$の余因子行列を$\tilde{A}$とすると、以下の関係が成立する。
    \begin{center}
        $A\tilde{A}=\tilde{A}A=dE\left (d=\rm{det}\left(A\right)\right)$
    \end{center}
\end{itembox}
\subsubsection{固有値と固有ベクトル (P.98)}
\subsubsection{行列の対角化}
正方行列$A$が与えられたとき、$B=P^{-1}AP$が対角行列になるような生息行列$P$と対角行列$B$を
求めることを行列$A$の対角化という。
\begin{itembox}[l]{固有多項式}
    正方行列$A$に対し、以下の多項式$g_A\left(t\right)$を$A$の固有多項式という。
    \begin{center}
        $g_A\left(t\right) = \left|tE-A\right| $
    \end{center}
\end{itembox}
\subsubsection{対角化可能性}
正方行列$A$は常に対角化されるとは限らない。
\begin{itembox}[l]{対角化の条件}
    $A$が$N$次の正方行列のとき、「$A$が対角化可能」とは
    $A$が$N$個の独立な固有ベクトルを持つことと同値である。
\end{itembox}
\section{微積分}
\subsection{多変数関数}
\subsubsection{偏微分}
多変数関数の微分 → 偏微分
\subsubsection{多変数関数の極値}
\begin{itembox}[l]{定理 4.3.3 極値を持つ必要条件}
    $f\left(x,y\right)$ が $\left(a,b\right)$で極値をとるならば、
    $f_x\left(a,b\right)=f_y\left(a,b\right)=0$ である。
\end{itembox}
\begin{itembox}[l]{定理 4.3.4 極値の判定}
    $f\left(x,y\right)$は$C^2$の関数であり、点$\left(a,b\right)$において
    $f_x\left(a,b\right)=f_y\left(a, b\right)=0$であるとする.\\
    判別式を、$D=f_{xx}\left(a,b\right)f_{yy}\left(a,b\right)-f_{xy}(a,b)^2$と定義する。
    \begin{enumerate}
        \item $D>0$とする。\\
              $f_{xx}\left(a,b\right)>0$ならば、$f$は点$\left(a,b\right)$で極小値をとる。\\
              $f_{xx}\left(a,b\right)<0$ならば、$f$は点$\left(a,b\right)$で極大値をとる。
        \item $D<0$ならば、$f$は点$\left(a,b\right)$で極値をとらない。
    \end{enumerate}
\end{itembox}

\subsubsection{接線の方程式}
\begin{itembox}[l]{定理 4.4.1 陰関数の定理}
    $f\left(x,y\right)$が$C^1$級の関数で、$f_x\left(a,b\right)=0$, $f_y\left(a,b\right)\neq0$ならば、\\
    $a$を含む開区間で定義された$f(x,y)=0$の陰関数$y=\phi\left(x\right)$で、$\phi\left(a\right)=b$となるものが存在する。\\
    このとき、$y=\phi\left(x\right)$は微分可能で、次の式が成り立つ。
    \begin{center}
        $\phi'\left(x\right)=\dfrac{f_x\left(x,\phi\left(x\right)\right)}{f_y\left(x,\phi\left(x\right)\right)}$,
        すなわち、$\dfrac{dy}{dx}=-\dfrac{f_x\left(x,y\right)}{f_y\left(x,y\right)}$
    \end{center}
\end{itembox}
\begin{itembox}[l]{例題 }

\end{itembox}

\end{document}