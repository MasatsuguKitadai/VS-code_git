\documentclass[a4paper]{jsarticle}
\setlength{\topmargin}{-20.4cm}
\setlength{\oddsidemargin}{-10.4mm}
\setlength{\evensidemargin}{-10.4mm}
\setlength{\textwidth}{18cm}
\setlength{\textheight}{26cm}

\usepackage[top=15truemm,bottom=25truemm,left=20truemm,right=20truemm]{geometry}
\usepackage[latin1]{inputenc}
\usepackage{amsmath}
\usepackage{amsfonts}
\usepackage{amssymb}
\usepackage[dvipdfmx]{graphicx}
\usepackage{listings}
\usepackage{listings,jvlisting}
\usepackage{geometry}
\usepackage{framed}
\usepackage{color}
\usepackage[dvipdfmx]{hyperref}
\usepackage{ascmac}

\hypersetup{
	colorlinks=false, % リンクに色をつけない設定
	bookmarks=true, % 以下ブックマークに関する設定
	bookmarksnumbered=true,
	pdfborder={0 0 0},
	bookmarkstype=toc
}

\lstset{
basicstyle={\ttfamily},
identifierstyle={\small},
commentstyle={\smallitshape},
keywordstyle={\small\bfseries},
ndkeywordstyle={\small},
stringstyle={\small\ttfamily},
frame={tb},
breaklines=true,
columns=[l]{fullflexible},
xrightmargin=0zw,
xleftmargin=3zw,
numberstyle={\scriptsize},
stepnumber=1,
numbersep=1zw,
lineskip=-0.5ex
}



\author{}
\title{数学}
\date{}

\begin{document}
\maketitle

\section{微積分}
\subsection{多変数関数の極値}
\begin{itembox}[l]{定理 4.3.3 極値を持つ必要条件}
    $f\left(x,y\right)$ が $\left(a,b\right)$で極値をとるならば、
    $f_x\left(a,b\right)=f_y\left(a,b\right)=0$ である。
\end{itembox}
\begin{itembox}[l]{定理 4.3.4 極値の判定}
    $f\left(x,y\right)$は$C^2$の関数であり、点$\left(a,b\right)$において
    $f_x\left(a,b\right)=f_y\left(a, b\right)=0$であるとする.\\
    判別式を、$D=f_{xx}\left(a,b\right)f_{yy}\left(a,b\right)-f_{xy}(a,b)^2$と定義する。
    \begin{enumerate}
        \item $D>0$とする。\\
        $f_{xx}\left(a,b\right)>0$ならば、$f$は点$\left(a,b\right)$で極小値をとる。\\
        $f_{xx}\left(a,b\right)<0$ならば、$f$は点$\left(a,b\right)$で極大値をとる。
        \item $D<0$ならば、$f$は点$\left(a,b\right)$で極値をとらない。
    \end{enumerate}
\end{itembox}

\subsection{接線の方程式}
\begin{itembox}[l]{定理 4.4.1 陰関数の定理}
$f\left(x,y\right)$が$C^1$級の関数で、$f_x\left(a,b\right)=0$, $f_y\left(a,b\right)\neq0$ならば、\\
$a$を含む開区間で定義された$f(x,y)=0$の陰関数$y=\phi\left(x\right)$で、$\phi\left(a\right)=b$となるものが存在する。\\
このとき、$y=\phi\left(x\right)$は微分可能で、次の式が成り立つ。
\begin{center}
    $\phi'\left(x\right)=\dfrac{f_x\left(x,\phi\left(x\right)\right)}{f_y\left(x,\phi\left(x\right)\right)}$,
すなわち、$\dfrac{dy}{dx}=-\dfrac{f_x\left(x,y\right)}{f_y\left(x,y\right)}$
\end{center}
\end{itembox}
\begin{itembox}[l]{例題 }
    
\end{itembox}

\end{document}