\documentclass[12pt,a4paper]{jsarticle}

\usepackage[latin1]{inputenc}
\usepackage{amsmath}
\usepackage{amsfonts}
\usepackage{amssymb}
\usepackage[dvipdfmx]{graphicx}
\usepackage{listings}
\usepackage{listings,jvlisting}
\usepackage{geometry}

\lstset{
basicstyle={\ttfamily},
identifierstyle={\small},
commentstyle={\smallitshape},
keywordstyle={\small\bfseries},
ndkeywordstyle={\small},
stringstyle={\small\ttfamily},
frame={tb},
breaklines=true,
columns=[l]{fullflexible},
xrightmargin=0zw,
xleftmargin=3zw,
numberstyle={\scriptsize},
stepnumber=1,
numbersep=1zw,
lineskip=-0.5ex
}

\author{来代 勝胤}
\title{モンテカルロ法による面積計算}

\begin{document}
\maketitle
\thispagestyle{empty}
\clearpage
\addtocounter{page}{-1}

\newgeometry{left=15mm,right=15mm,top=15mm,bottom=20mm}

\begin{flushleft}
    {
        \LARGE \textbf{}
    }
\end{flushleft}

\section{理論}

以下の5つの曲線と、$x=0$、$x=1$、$y=0$が囲む面積を求める。

\begin{eqnarray}
    f(x)=&x\\
    f(x)=&x^2\\
    f(x)=&\cos\left(\frac{\pi}{2}x\right)  \\
    f(x)=&\sin\left(\pi x\right) \\
    f(x)=&\exp\left(-x^2\right)
\end{eqnarray}

乱数を用いたモンテカルロ法で面積を評価する。
$n$回目に発生させた乱数を$r^n$として、設定点座標を$\left(r^n, r^{n+1}\right)$で与える。\\
\begin{center}
    設定点の$y$座標          $r^{n+1}$\\
    設定点の$x$座標の関数値 $f\left(r^n\right)$
\end{center}
上記、2式の大小を比較し、$r^{n+1}>f(r^n)$であれば、$S_1$領域、$r^{n+1}<f(r^n)$であれば$S_2$領域に属するものと判断できる。
ここで、設定点を$N$点としたときに、今回の場合においては$S_2$領域に属する設定点の比率から面積を求めることが可能である。\\
\newpage
\small
\begin{lstlisting}
    !Program Name : report.f90

    program main
    
        implicit none
        integer, parameter :: num = 10000
        integer, parameter :: case = 5
        integer i, j, n
        real x, y, a, b, pi
        real f(case),s(case)
        real r(num + 1)
        real area(num + 1,case)
    
        !** create random number *********************
        
        integer :: seedsize
        integer, allocatable :: seed(:)
        
        call random_seed(size = seedsize)       
        allocate(seed(seedsize))               
        do j = 1, seedsize
            call system_clock(count = seed(j)) 
        end do
        call random_seed(put = seed(:))         
        
        !********************************************
    
        pi = 4.0 *atan(1.0)
        ! write(6,*)'pi = ',pi
    
        call random_number(r)
    
        do 10 j = 1 ,case
           s(j) = 0
        10 continue
    
        open(17, file = 'output/result.dat', status='replace')
    
        do 20 i = 1, num
            x = r(i)
            y = r(i + 1)
        
            f(1) = x                  !case_1
            f(2) = x*x                !case_2
            f(3) = cos(pi / 2.0 * x)  !case_3
            f(4) = sin(pi * x)        !case_4
            f(5) = exp(-x * x)        !case_5
    
            do 100 j = 1, 5   
                if (f(j) < y) then
                    s(j) = s(j) + 1
                end if    
                a = (s(j)/i)
                area(i,j) = a
            100 continue
            20 continue
    
    do 30 i = 1, num
        write (17,*) i,area(i,1),area(i,2),area(i,3),area(i,4),area(i,5)
    30 continue
    
    close(17)
    
    
    end program
\end{lstlisting}

\end{document}